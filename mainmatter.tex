\section{Introduction}\label{sec:intro}
zbMATH classified more than 
\href{https://zbmath.org/?q=%2A+py%3A2019}%
{80k} articles in 2019 according to the Mathematical Subject Classification~(MSC).
With more than
\href{https://msc2020.org}%
{5000} class labels, this classification task requires significant depth-knowledge of the particular fields of mathematics in order to obtain correct fine-grained classifications of the articles.
Therefore the classification is two-fold.
At first, the articles are pre-classified into one of \href{https://msc2020.org}%
{63} main classifications, which are the two first digits of the first MSC label of each article.
In a second step, domain expert assigns fine-grained classification labels in their area of expertise.

In this paper, we focus on the coarse-grained classification and explore how modern machine learning technology can be employed to automate this process.
In particular, we compare the current state of the art technology with a system customized for the application in zbMATH from 2014~\cite{SchonebergS14}. 
We formulate the following research questions:
\begin{enumerate}
  \item Which measures are appropriate to measure the quality of automatic classifications?
  \item Does the inclusion of mathematical formulae improve the quality of the classifications?
  \item Does the Part of Speech (POS) preprocessing improve the quality of classifications?
  \item Which features (Title, Abstract, References) are most important for the correct classification?
  \item How does the quality of automatic classification compare to manual classifications?
  \item How to avoid misclassifications in the future?
  \item How can the best performing method be integrated into the current production pipeline at zbMATH?
\end{enumerate}
\section{Method}\label{sec:method}

\section{Evaluation}\label{sec:eval}

\section{Conclusion \& Future Work}\label{sec.concl}

\printbibliography[keyword=primary]